% figures in chp.tex files need to have paths relative to main.tex
\documentclass[a4paper]{article}

\usepackage{a4}
\usepackage{epsfig}
\usepackage{graphics,graphicx}
\usepackage[lmargin=4cm,rmargin=2cm,tmargin=3.5cm,bmargin=3.5cm]{geometry}
\usepackage{natbib}
\bibpunct{(}{)}{;}{a}{,}{,}
\usepackage{setspace}
\usepackage{amsmath}
\usepackage{amssymb}
\usepackage{amsthm}
\usepackage{amsfonts}
\usepackage{array}
\usepackage{tabulary}
\usepackage{txfonts}
\usepackage{mathtools}
\usepackage{upgreek}
\usepackage{subfig}
\usepackage{float}
\usepackage{longtable}
  \newtheorem{lem}{Lemma}
  \newtheorem{thm}{Theorem}
  

\makeatletter

\newcommand\bappendix{%
  \addtocontents{toc}{\protect\addvspace{10pt}Appendices}
  \begingroup
  \renewcommand\section{\stepcounter{appendix}%
    \renewcommand\thesection{\Alph{appendix}}
    \@startsection {section}{1}{\z@}%
      {-3.5ex \@plus -1ex \@minus -.2ex}%
      {2.3ex \@plus.2ex}%
      {\normalfont\Large\bfseries}}}
\makeatother
\newcommand\eappendix{\endgroup}

%\usepackage{lineno}
%\setcounter{page}{1}

%\pagenumbering{arabic}

\usepackage{fancyhdr}

\newcommand{\rr}{\raggedright}
\newcommand{\tn}{\tabularnewline}
\newtheorem{mydef}{Definition}

\begin{document}

\title{Trade-offs involving juvenile competition offer more robust coexistence than a fecundity-defence trade-off in communities dependent on disturbance for coexistence}
\maketitle

\section*{Abstract}
The life history of plants vary on at least three important axes; fecundity, juvenile growth and defence against herbivory and/or abiotic disturbance such as fire. We use a time homogeneous Markov Chain model to enumerate the likelihood of coexistence for a variety of trade-offs along these three axes, and consider how robust to parameter changes this likelihood is. We show; (a) how disturbance is measured (frequency or intensity) has a significant effect on the observed community response, (b) when juvenile competition through rapid growth is traded off against other life history traits, coexistence is robust to changes in system capacity. However, when fecundity and defence against disturbance are traded off, increases in system capacity dramatically reduce the likelihood of coexistence, and; (c) A trade-off between fecundity and juvenile growth rates is most likely to contribute to coexistence in diverse communities experiencing disturbance.

\vspace{.5cm}
keywords: lottery model; community ecology; life history; fire ecology; diversity

\section{Introduction}\label{intro}
Several previous studies \citep[e.g.][]{denslow1987tropical,sousa1984role} have suggested that disturbance events may play an important part in promoting diversity in some systems. Disturbance, that is, an event that results in the death of large number of individuals and alter niche opportunities \citep{shea2004moving}, creates an environment that prevents the exclusion of weaker competing species, such as pioneer species. One such disturbance event is fire, which significantly shapes vegetative structure and composition \citep[e.g.][]{debano1998fire,hart2005post}, with much fire management being used to achieve ecological goals such as maintaining key habitats \citep{avitabile2013systematic}. Vegetation shows ecological adaption in order to survive fire disturbances \citep[e.g.][]{agee1996fire,keeley2011fire}. These adaptions can vary significantly, and plant species also differ in other life history traits. The three important life history traits are reproduction, growth from seedling to adult, and defence against both herbivory and abiotic factors such as fire \citep{bazzaz1987allocating}. Trade-offs between these three traits may account for the high levels of diversity observed in natural systems. Theory has shown that these trade-offs can allow two or more species to coexist while competing for  the same resources in an environment \citep[e.g.][]{kisdi2003coexistence,levins1971regional,bonsall2004life}, suggesting that trade-offs are important for sustaining high levels of biodiversity in nature, although this coexistence may be unlikely or unstable \citep{nattrass2012quantifying,gyllenberg2005impossibility}. However, while some studies have compared several trade-offs \citep[e.g.][]{tilman1990constraints,grime1977evidence}, little is understood about how multiple trade-offs combine to affect community diversity, and which trade-offs are most likely to contribute to coexistence.

Further, it has been argued that there are at least three different axes upon which disturbances should be measured: frequency; area affected; and intensity \citep{malanson1984intensity,miller1982community,sousa1984role}.  However, many empirical studies consider disturbance levels in forests as a single parameter, such as the total area lost in a given time period (effectively frequency multiplied by intensity) \citep[e.g.][]{molino2001tree,nakagawa2000impact,peterson1997tornado}. Fire ecology studies have often acknowledged the difference between different factors influencing fire, such as ignition probability and spread \citep[e.g.][]{kilgore1979fire,turner1994landscape}, observing that most fires are of low intensity and high frequency \citep{kilgore1979fire}, while spatial connectedness has an impact on overall fire intensity \citep{turner1994landscape}. However, there remain relatively few theoretical studies that consider the relative impact of frequency and intensity of disturbances \citep[but see ][]{miller2011frequency}. \cite{miller2011frequency} show how disturbance intensity and frequency have different effects on community dynamics and the community response depends greatly on the combination of the two. It seems likely that this may be a significant factor behind the relative lack of consensus on the effects of altered disturbance regimes in nature communities.


Here, we compare the likelihood of pairwise trade-offs between fecundity, growth and defence against disturbance sustaining two species with identical resource requirements. Defence is taken to mean the ability of an individual to withstand a disturbance event, that is, an event that results in the death of large number of individuals and alters niche opportunities within the community, while seed production (seed number per capita per year) is used as the measure of fecundity. The probability of species disturbance intensity parameters resulting in coexistence is calculated, and we demonstrate that a fecundity-growth trade-off gives the greatest likelihood of coexistence. Trade-offs involving growth differences are robust to changes in community size, while a fecundity-defence trade-off cannot support coexistence in large systems for any expected fire disturbance intensity. We also demonstrate that changes in frequency and intensity can affect the community response to disturbance different, and produce varied disturbance-diversity relationships (DDRs).

\section{The Model}
\label{model}
We consider a lattice of a number $M$ of fixed sites with size $C$. Each site can contain one adult individual of one of two species present in the community, or in the absence of an adult individual, may contain juveniles. These sites are observed in continuous time, and at times $\tau_1,\tau_2,\dots$ there is an adult death, with the individual randomly selected from a uniform distribution, such that each adult present in the community has an equal chance of mortality. These death events are followed by competition among juveniles to claim the site that previously contained the dead adult. We assume
\begin{enumerate}
\item The death of adults and the growth of successful juveniles occur in a ratio of 1:1, such that the number of adults present $N$ is constant when considered at times $\tau_t,t=1,2,\dots$. This enables us to consider only the population dynamics of one species, with the population of the second species determined by the relationship $N_1(t) + N_2(t) = N$, where $t$ represents time as measured by the number of death events that have occurred, and $N_i(t)$ represents the population of species $i$ is the population after the $t$-th death event. We therefore consider the dynamics only of the species 1 population $n$, with $N_1(t)=n, N_2(t)=N-n$.
\item The two species considered are identical in all bar two aspects. They differ only in annual per capita seed production $s_i$ and juvenile growth rates $g_i$. These juvenile growth rates are considered to be a linear growth from a seed of size $0$ to an adult occupying the entire site, i.e. of size $C$. We assume without loss of generality species 2 is the faster growing species, $g_2>g_1$.
\item Both species are assumed to be asexual or self fertilizing, such that a single individual can produce seeds and therefore offspring.
\item Following the death of an adult individual, the remaining $N-1$ individuals produce seeds that can colonise the empty site, or gap. Seeds cannot survive in the understorey, and die immediately. Therefore, when an adult dies, the gap created is considered to be empty until new seeds arrive at that site.
\item All seeds that arrive in a gap are considered viable, and in the absence of competition would successfully reach adulthood. They are produced in time according to a Poisson process with $\lambda_i =  s_i N_i$ the average number of seeds to be produced per unit time.
\item The seeds produced are spread uniformally between sites. After a death event, the expected number of seeds from species $i$ to arrive per unit time at the gap created is given by $s_i \hat{N}_i /N$, where $\sum_{i=1}^2 \hat{N}_i =N-1$ due to the death of an individual.
\item Once seeds arrive in an site not containing any adults, they instantly begin to grow at rate $g_i$, there is no latency period. When all the juveniles in a given site combine to occupy the entire gap, that is when the sum of their sizes is equal to $C$, then the smallest of these juveniles will die while the others continue to grow. This continues until there is only a single individual remaining, which will then occupy the site to adulthood. Note that since there is no intraspecific variation, this allows us to only consider the first seed of each species to arrive at a site.
\item Only considering the first seeds of each species to arrive in a gap, colonisation of a gap becomes a race to colonise and grow to size $C/2$, and any other individuals present will have size less than $C/2$ and will therefore be killed when in competition wit the larger juvenile.
\end{enumerate}
These assumptions allow us to model the system as a time homogeneous Markov Chain model where states are defined by the number of species 1 adults $n$, each time step is the replacement of an individual adult that dies, and the replacement of that individual is determined by the present populations of each species present in the community. This, the number of species 1 individuals $n$ will increase if an individual of species 2 dies, species 1 seeds arrive first and reach size $C/2$ before any subsequent species 2 seeds that arrive in the site reach that height. A seed of species 2 cannot reach the height $C/2$ before species 1 if it arrives in the site when species 1 has been present for a time greater than $x=C(1/g_2-1/g_1)/2$. As both species distribute seeds randomly throughout the environment, the probability of species 1 increasing in number is given by
\begin{align}
\label{increase}
p_{n,n+1}=&P(N_1(1)=n+1)|N_1(0)=n) \\
&=\frac{N-n}{N}\frac{s_1 n}{s_1 n +s_2(N-n-1)}\exp \left( -s_2 \frac{N-n-1}{N} x\right),\notag \end{align}
where the exponential term is the probability that no seeds of species 2 arrive in time $x$. Similarly, species 1 will decline in numbers if it suffers a death, and species 2 claims the site, either arriving at the site first or by successfully invading a site where species 1 juveniles are present, giving
\begin{align}
\label{decrease}
p_{N-1,N}=&P(N_1(1)=n-1 | N_1(0)=n)  \\ 
=&\frac{n}{N}\frac{s_2 (N-n)}{s_1 (n-1) +s_2(N-n)} \notag \\
& + \frac{n}{N}\frac{s_1 (n-1)}{s_1 (n-1) +s_2(N-n)}\left(1-\exp \left( -s_2 \frac{N-n}{N} x\right)\right). \notag \end{align}
All other possibilities result in the populations of the two species remaining the same after a time step, so we use a third function $p{n,n}=1-p_{n,n+1}-p_{n,n-1}$ to fully define the transition probabilities, giving a $(N+1) \times (N+1)$ tridiagonal transition matrix $A$, with $A_{i,i}=p_{n,n}, A_{i,i+1}=p_{n,n+1}, A_{i,i-1}=p_{n,n-1}$. Note that these rules ensure the system is closed, with no seeds arrived from outside the system, there therefore that the probability of moving from a state where a species is absent to a state where it is present equals zero. Mathematically, this gives a system with absorbing boundaries when either species is extinct; $n \in \{0,N\}$.

We also consider disturbance events that occur during the time step $t \to t+1$ with probability $f=(dNT_D)^{-1}$ where $T_D$ is the expected number of years between disturbance events, and $d$ is the expected annual mortality rate (fixed at $1\%$). Within these events, individuals die with probability $I$ (the intensity of the disturbance), before the remaining individuals compete to fill the emptied sites. Increasing either frequency $f$ or intensity $I$ will result in an increase in total basal area lost over a given time frame, a standard measure of disturbance \citep[e.g.][]{molino2001tree,peterson1997tornado}. During a disturbance event, each individual has a probability $I$ of mortality, resulting in $d_{dist}=d_1+d_2$ total deaths, where $d_i$ is the number of species $i$ deaths. The remaining $N-d_{dist}$ individuals compete over the empty site, with species 1 colonising if it reaches the site at least time $x$ before species 2. Therefore, species 1 will take a site with a probability defined by the function
\begin{equation}
\label{sp1}
Sp_1(n_1^*,n_2^*)=\frac{s_1 n_1^*}{s_1n_1^*+s_2n_2^*}\exp \left(-s_2 x\frac{n_2^*}{N}\right),
\end{equation}
where $n_1^*,n_2^*$ are the populations remaining immediately after the deaths from a disturbance event.

We mostly consider the case where each adult individual is reproductively active at all times, but in real communities, the proportion of adults that are reproductively active can vary significantly. Therefore, we also consider the case where in each time step, each individual is producing seeds with probability $p$, which has the effect of reducing the productivity of the system. This parameter $p$ is assumed to be time invariant and independent of species abundance, as well as constant across species. This allows us to consider the effects of seed limitation, which we compare with other ways of altering productivity. 


\section{Conditions for Coexistence}
To determine whether two species coexistent in the model, we use a version of stochastic boundedness. Chesson (1982) demonstrated that if the average growth rates at the boundaries of a lottery model with infinite sites are positive, the model is stochastically bounded, and both species coexist. We therefore consider the expected change when a species is extremely rare, with a single individual present (i.e. $n=1,N-1$), which we approximate using the function
\begin{equation}
\label{avchange}
\begin{array}{rl}
\text{AverageChange}(n) = (1-f)\left[p_{n,n+1}-p_{n,n-1}\right]&\\[0.75em]
+ f\left[ NI Sp_1(n(1-I),(N-n)(1-I)) - nI\right]&.
\end{array}
\end{equation}
The first term in \eqref{avchange} is the expected change when disturbance does not occur multiplied by the within-event probability that there is no disturbance event. The second term indicates the expected change during a disturbance as measured by the expected remaining populations. Due to Jensen's inequality, this is not the actual expected change, but an approximation. This approximation improves as system capacity $N$ is increased, as demonstrated in Appendix~\ref{app2a}, and works well for $N\geq500$. Coexistence is predicted for regions of parameter space where the following 2 inequalities both hold;
\begin{align}
\label{ac1}\text{AverageChange}(1)&>0, \\
\label{acn-1}\text{AverageChange}(N-1)&<0. \end{align}
Note that when there is no disturbance ($T_D=\infty$), these conditions simplify to 
\begin{align}
\label{lowerboundarycond}p_{1,2} &>p_{1,0} \\
\label{upperboundarycond}p_{N-1,N} &<p_{N-1,N-2}. \end{align}
In biological terms, these conditions can be considered as the following: If there is a single individual of a species remaining, for that species to persist it must produce offspring that successfully colonise at least one other site before the adult dies. We show in Figure~\ref{confidenceints} and Appendix~\ref{app:stationary} that this is a good approximation of coexistence observed within time series simulations.

Unlike the work of Chesson (1982), we are modelling finite populations, rather than considering the proportion of sites occupied by a species. Therefore,  the boundary conditions are considered with a single individual present, rather than being evaluated at 0 as in Chesson's work. This combined with the non zero probability of a single individual dying at a given time ensures that the only stable distribution for the model as $t \to \infty$ is $P(N_1(t)=0) = P(N_1(t)=N) = 1/2.$ That is, the coexistence predicted by stochastic boundedness in a closed system is unstable. Despite this, as demonstrated by Figure~\ref{confidenceints}, times series simulations do demonstrate coexistence on ecological time scales. Further, as system capacity $N$ increases, the model is well approximated by considering the proportion of sites $Q_i$ containing species $i$. Then, performing an analysis as in Chesson (1982) we can return the conditions for coexistence given above when considering the asymptotic growth rate of a species with zero density. At lower system capacity, a low level of immigration from elsewhere can give a fixed distribution where species can coexist, and as we show in Appendix~\ref{app:stationary}, a very low level of immigration produces a fixed, stable distribution that is maximised at an internal point $x \in [2,N]$ such that $\pi_x > \pi_{1}, \pi_{N+1}$ where $\pi_i$ is the probability of having $i-1$ species 1 individuals at the stable distribution) with approximately the same conditions as the stochastic boundedness criteria outlined above.

\begin{figure}[htbp]
\begin{center}
\begin{tabular}{cccc}
(a)&&(b)&\\
&\includegraphics[width=2.5in]{lowxCIplot.pdf} && \includegraphics[width=2.5in]{medxCIplot.pdf}\\
(c)&&&\\
&\includegraphics[width=2.5in]{highxCIplot.pdf}&& \end{tabular}
\caption{(a) Time series simulations satisfying the conditions \eqref{lowerboundarycond} but not \eqref{upperboundarycond} (x=0.01) show extinction of the more fecund species 1. (b) When both conditions are satisfied (x=0.035), both species will coexist in the environment over ecological time scales. (c) When \eqref{upperboundarycond} is satisfied yet \eqref{lowerboundarycond} is false (x=0.06), species 2 will go extinct. Lines show the mean and 95\% confidence intervals of 1000 simulations in each scenario. Parameters $s_1=500,s+2=50, N=1000$.}
\label{confidenceints}
\end{center}
\end{figure}

We determine the  likely contribution to coexistence of a trade-off by considering the region of $I_1 - I_2$ space for which coexistence is predicted by the conditions \eqref{ac1} and \eqref{acn-1}. Trade-offs where a large proportion of parameter space predicts coexistence are considered more likely to contribute to the observed diversity in empirical communities. It is biologically unlikely that an increase in disturbance intensity experienced by one species will be matched by a decrease in the intensity disturbed by the second species. For example, if a fire increases in temperature, it is expected that all species present will experience an increase in mortality. To account for this, we link species specific intensities in a simple, linear manner $I_1=yI_2$. The size of the intensity range that predicts coexistence is then given by $I_2^* - I_2^+$, where $I_2^*$ is the intensity at which $AverageChange(N-1)=0$, while $I_2^+$ satisfies $AverageChange(1)=0$. \textbf{This is then scaled by the range of $I_2$ values for which both species can survive in monoculture, i.e. $\min(1,1/y)$, such that $P(coexist)=(\min(1/y,I_2^*) - \min(1/y,I_2^+))/\min(1,1/y)$, where $\min(1,1/y)$ is used to replace the calculated root if the root falls outside the possible range of intensities.} For trade-offs including species specific growth rates, the roots are calculated using the FindRoot function in Mathematica 8. However, for a fecundity-defence trade-off, the absence of an exponential term allows the explicit calculation of the roots $I_2^*,I_2^+$, as outlined in Appendix~\ref{app:fdtoroots}.

\section{Results} \label{results}
\subsection{Likelihood of coexistence}
In the absence of disturbance, only the trade-off between fecundity and juvenile growth ($y=1, I_1=I_2=I$ can give coexistence of two species, when the time required for preemption $x$ satisfies
\begin{equation}
\label{xrange}
x_{min}=\frac{N\ln\left(\frac{s_1(N-1)}{s_1(N-2)+s_2}\right)}{s_2}<x<\frac{N\ln\left(\frac{s_1(N-1)}{s_1+s_2(N-2)}\right)}{s_2(N-2)}=x_{max}.
\end{equation}
Note $x_{min},x_{max}>0$ for the fecundity-growth trade-off, with $s_1>s_2, g_1<g_2$. Note also that as system capacity $N$ is increased, this range of $x$ asymptotes to
\begin{equation}
\label{bignxrange}
\frac{1}{s_2}-\frac{1}{s_1}<x<\frac{\ln\left(\frac{s_1}{s_2}\right)}{s_2}.
\end{equation} 
When the time required for preemption is small, $x<x_{min}$, the more fecund species 1 will exclude species 2. This holds when disturbance regimes are included, as the creation of multiple gaps increases the niche opportunities for the more fecund species, at the expense of the strong local competitor. When $x$ satisfies the conditions in \eqref{xrange}, low intensity or infrequent disturbances maintain coexistence, but disturbances with high or intermediate intensity will reduce the diversity by excluding the strong local competitor. This range of intensities that give coexistence increases in size monotonically with $x<x_{max}$, with all intensities $I \leq 0.695$ exhibiting coexistence for $s_1=500,s_2=50,N=1000,x=x_{max}=10 \ln(555/56)/499,T_D=10$.

More interesting is the case $x>x_{max}$, where disturbance is necessary for coexistence, as with trade-offs between disturbance-defence and fecundity or growth. In this region, as $x$ increases, the range of disturbance intensities $I$ that give coexistence decline. However, as shown in Figure~\ref{fig:comparison}, for $x>x_{max}$ coexistence is much more likely for the fecundity-growth trade-off than the trade-offs including defence against disturbance for a fixed frequency of disturbance. Allowing species to vary along all three axes does not significantly increase the likelihood of coexistence over that of the fecundity-growth trade-off.
\begin{figure}
  \includegraphics[width=4.5in]{Systemsizechanges.pdf}
   \caption{How the probability of coexistence varies with system capacity $N$ for trade-offs between fecundity and growth (blue) ($x>x_{max}$), growth and defence (black) and fecundity and defence (pink, dashed), when disturbance frequency is fixed. The fecundity-growth trade-off is much more likely to support coexistence than trade-offs including defence. Trade-offs including juvenile growth differences are robust to changes in system capacity, with the probability of coexistence tending to a positive value as $N \to \infty$. However, a fecundity-defence trade-off cannot support two species for large communities, with the likelihood of coexistence declining to zero as $N$ goes to infinity. Parameters used; Fecundity-defence trade-off $s_1=500,x=0,y=2$; Growth-defence trade-off $s_1=50,x=0.06,y=2/3$; Fecundity-growth trade-off $s_1=500,x=0.06,y=1$; All cases $s_2=50, T_D=10$.}
 \label{fig:systemsize}
\end{figure}
The trade-off between fecundity and defence against disturbance reacts differently to changes in system capacity $N$ than trade-offs involving juvenile growth. Both fecundity-growth and growth-defence trade-offs increase with $N$ initially, but the probability of coexistence asymptotes to a fixed value as $N$ increases towards infinity. In contrast, as $N$ increases in the fecundity-defence case, the root $I_2^*$ increase above $1/y$, the point at which coexistence is not possible as $I_1 \geq 1$, which causes species 1 to go extinct within the first disturbance event. In this case, we define coexistence as $\min(1/y,I_2^*) - \min(1/y,I_2^+)$, and as $N$ increases above the point at which $I_2^*=1/y$, the probability of coexistence begins to decrease, tending to zero as the system capacity gets very large.

However, a trade-off between defence and fecundity is more robust to changes in the parameter $y$ than a growth-defence trade-off. Coexistence is possible for any $y>1$ along a fecundity-defence trade-off, with the probability of coexistence increasing slightly with $y$, reaching a maximum as $y \to \infty$, such that $I_2=0$ for all $I_1 \leq 1$. However, as $y<1$ is decreased in the growth-defence case, there is a threshold value $y^*$ at which $I_2^*=I_2^+$ For $y<y^*$, invasion by either species is not possible, instead giving a region of $I$- space where founder control occurs. The value of $y^*$ decreases as the time required for preemption $x$ increases, and has the value $y^* \approx 0.185$ for parameters $s_1=s_2=50,x=0.06,N=1000,T_D=10$.

\subsection{Robustness of coexistence to altered parameters}
A fecundity-defence trade-off also exhibits sensitivity to changes in seed numbers. For fixed $s_2$ as $s_1>s_2$ increases, there is initially a rapid increase in the likelihood of probability. However, a threshold is reached when $I_2^*$ increases above $1/y$. At this point, the likelihood of coexistence decreases, and as $s_1$ tends to infinity, the likelihood of both species coexisting tends to zero. For $s_2=50,T_D=10,y=2$, the maximum likelihood of coexistence occurs at approximately $s_1=890$. Increasing productivity, the production rate of new biomass, in this case by altering seed numbers for both species by a factor $\phi$, does not have any effect on the likelihood of coexistence. For all parameters considered, the maximum probability of coexistence is approximately 0.2.

When the trade-off is between growth and defence, the likelihood of coexistence is maximised when the time for which the slower growing species must occupy a site uncontested to claim the site $x$ is intermediate. For low values of $x$, the likelihood of coexistence increases with the difference in species growth rates, b,efore declining as the difference in growth rates becomes large. For $s_1=s_2=50,T_D=10,y=2/3$, the maximal coexistence is given by $x \approx 0.203$, where $P(coexist)\approx 0.136$. At this point,  A fecundity-growth trade-off exhibits a similar response to changes in $x$, with the probability of coexistence peaking at intermediate values.

When productivity is altered in the fecundity-growth or growth-defence cases, the effects depend on how productivity is included in the model. Productivity, the production rate of new biomass, can be in the form of increased seed numbers (fecundity), increased juvenile growth rate, or both. Multiplying juvenile growth rates by a factor $\zeta$, such that the growth rate of species $i$ becomes $\zeta g_i$ with lead to the transformation $x \to x/\zeta$. Therefore, an increase (decrease) in growth rates will result in a decrease (increase) in the time $x$ the slower growing species must be present in an uncontested site before it can overcome seedlings of the faster growing species 2. The community response to this change therefore depends on the value of $x$ initially. If $x$ is low, an increase in growth rates will decrease the likelihood of coexistence, but for species with large differences in growth rates (and therefore large $x$) an increase in growth rates will increase the likelihood of coexistence.

In contrast, if seed numbers are multiplied so an individual of species $i$ produces $\phi s_i$ seeds per year, the effect is to increase the amount of time $x$ required for preemption to occur. The factor $\phi$ cancels from numerator and denominator of the colonisation probability \eqref{sp1}, leaving only the exponential term alters, the factor $-s_2 x$ becoming $-s_2 \phi x$. Note this is equivalent to multiplying $x$ by a factor $\phi$. Therefore, an increase (decrease) in seed production will mean an increase (decrease) in $x$, the opposite effect as when growth rate change. Therefore, an increase in seed production will improve the likelihood of coexistence when $x$ is small, but decrease coexistence probability when $x$ is large and species differ greatly in growth rates. Note that if both seed production and growth rates are effected by a change in productivity, $x$ will effectively transform to $x \phi / \zeta$. Note also that with a fecundity-growth trade-off, this can move $x$ into the range that gives coexistence in the absence of disturbance (Figure~\ref{fig:FvIwithprod}(g)).

\begin{figure}[th]
\centering
\begin{tabular}{lcccccc}
 &(a)&$\frac{\phi}{\zeta}=\frac{1}{3}$&(b)&$\frac{\phi}{\zeta}=1$&(c)&$\frac{\phi}{\zeta}=3$\\
 && \includegraphics[width=1.5in]{fdphizeta1over3.pdf} && \includegraphics[width=1.5in]{fdphizeta1.pdf} && \includegraphics[width=1.5in]{fdphizeta1over3.pdf} \\
 &(d)&&(e)&&(f)&\\
 && \includegraphics[width=1.5in]{gdphizeta1over3.pdf} && \includegraphics[width=1.5in]{gdphizeta1.pdf} && \includegraphics[width=1.5in]{gdphizeta3.pdf} \\
 &(g)&&(h)&&(i)&\\
 && \includegraphics[width=1.5in]{fgphizeta1over3.pdf} && \includegraphics[width=1.5in]{fgphizeta1.pdf} && \includegraphics[width=1.5in]{fgphizeta3.pdf} \\
 \end{tabular}
 Time between disturbances $\ln(yrs)$
   \caption{\textbf{Coexistence with productivity changes: }The region of coexistence (shaded area), with $x>x_{max}$ for a fecundity-growth trade-off, responds to a change in productivity where the number of seeds produced is multiplied by a factor $\phi$ and growth rates by a factor $\zeta$. (a-c) Productivity changes on a fecundity-defence trade-off have no effect on the region of coexistence. (d-f) On a growth-defence trade-off, low $\phi/\zeta$ (either faster growth rates or low seed production effectively reduces the time required for preemption $x$. This will increase the likelihood of coexistence for high $x$ (shown here) but cause a reduction in coexistence if $x$ is sufficiently low initially. High $\phi/\zeta$, caused by lowering growth rates or high seed production has the opposite effect. (g-i) On a fecundity-growth trade-off, when $x>x_{max}$, an increase in $\phi/\zeta$ ensures the system needs higher intensity disturbances to exhibit coexistence, and reduces the likelihood of coexistence. Lowering $\phi/\zeta$ mimics the effect of lowering the difference in growth rates, and increases coexistence. Note this can allow coexistence when disturbance does not occur ($T_D \to \infty$ or $I_2=0$ in (g)). Parameters used; Fecundity-defence trade-off $s_1=500,x=0,y=2$; Growth-defence trade-off $s_1=50,x=0.06,y=2/3$; Fecundity-growth trade-off $s_1=500,x=0.06,y=1$; All cases $s_2=50$.}
 \label{fig:FvIwithprod}
 \end{figure}

\subsection{Different effects of disturbance frequency and intensity}
All three trade-offs exhibit a similar response to changes in disturbance frequency (with $x>x_{max}$ in the case of the fecundity-growth trade-off), with the probability of coexistence peaking at intermediate disturbance frequency. However, the effects on community structure of changes in frequency is very different from the community changes when disturbance intensity varies, as demonstrated in Figures~\ref{fig:FvIwithN}~and~\ref{fig:hockeyTd}. 
\begin{figure}[th]
\centering
\begin{tabular}{lcccccc}
 &(a)&$\ln(N)=4$&(b)&$\ln(N)=7$&(c)&$\ln(N)=15$\\
 && \includegraphics[width=1.5in]{fdtoimageexp4.pdf} && \includegraphics[width=1.5in]{fdtoimageexp7.pdf} && \includegraphics[width=1.5in]{fdtoimageexp15.pdf}
 \end{tabular}
 Time between disturbances $\ln(yrs)$
   \caption{\textbf{Coexistence with disturbance: }The region of coexistence (shaded area) for $x>x_{max}$ for a fecundity-defence trade-off, and the effect of increasing system capacity. Likelihood of coexistence is maximised at intermediate system capacity. Parameters used;  $s_1=500,s_2=50,x=0,y=2$.}
 \label{fig:FvIwithN}
\end{figure}
Both trade-offs featuring juvenile growth rates exhibit similar behaviour. Either high frequency, medium intensity or intermediate frequency, high intensity disturbances can support two species (although the analytic predictions here are less accurate for high intensities (Appendix~\ref{app:simcompare}). These results are robust to changes in systems size, as the region of coexistence stabilises to a fixed region when $N$ increases. A fecundity-defence trade-off responds differently. For small system capacity $N$, only very high frequency, high intensity disturbance regimes can support coexistence (Figure~\ref{fig:FvIwithN}(a)). For intermediate $N$, patterns are similar to those observed with the trade-offs involving growth, although higher intensities (for the more susceptible species) are necessary for coexistence (Figure~\ref{fig:FvIwithN}(b)). As $N$ is increased further, coexistence requires much longer time periods between disturbances to maintain two species. In this case, exemplified in Figure~\ref{fig:FvIwithN}(c), many communities will experience disturbances too frequent for any range of intensities to support more than a single species.

Focussing on the fecundity-growth trade-off model, analysis shows that it is possible to get very differently shaped DDR curves in this model, and this is affected by which factor influencing disturbance is considered. Increasing intensity can give very different responses to altering the frequency of disturbances. If $f=(dNT_D)^{-1}$ is fixed such that $1/f<T$, where $T$ is the expected time to extinction, increasing intensity will give a humped DDR. For example, if $\ln(T_D)=1$ is fixed, and intensity increased from $I=0.1$, we initially see an increase in diversity as intensity crosses the point $I\approx0.25$. As intensity is increased further, however, there is a decline in diversity at $I\approx0.72$. This produces the classic peaked DDR predicted by the IDH.

However, if the fixed frequency is low, coexistence is impossible for any intensity $I$, and so a flat DDR is produced. Meanwhile, if intensity $I$ is fixed, while frequency of disturbance is varied, there are three possible scenarios: i) When $I$ is low (e.g. $I=0.1$), the stronger competitor, species 2, will dominate regardless of frequency ; ii) For intermediate intensities (e.g. $I=0.5$), increasing frequency will cause diversity to increase from one species to two (when $\ln(T_D)\approx 5$ for $I=0.5$), a monotonically increasing DDR ; iii) When intensity is high (e.g. $I=0.8$), increasing frequency will match the predictions of the IDH, giving a peaked DDR with coexistence occurring for intermediate frequencies (when $I=0.8$, the peak in diversity is predicted in the range $2\leq \ln(T_D) \leq 6$). However, this coexistence occurs for a narrow range of $f$ (Figure~\ref{fig:simulationdata} in Appendix~\ref{app:simcompare}).


\section{Discussion}
Both life history trade-offs and disturbance events have been suggested as mechanisms that can promote and support high levels of diversity in nature \citep[e.g.][]{adler2000space,denslow1987tropical,sousa1984role,turnbull1999seed}. Here, we demonstrate that trade-offs among the three major life history traits for plants can give coexistence of at least two species competing within the same niche, and numerically quantify the likelihood of coexistence for each trade-off. However, the effectiveness of these trade-offs in supporting multiple species varies dramatically. Where species do not differ in juvenile growth rates, only small communities can possibly support more than a single species, and even in these small communities coexistence is unlikely outside a narrow range of parameters. In contrast, we demonstrate that trade-offs between growth and fecundity or defence can support two species for any community size. Finally, we confirm previous theory \citep{miller2011frequency} by demonstrating that differences in how disturbance is measured (here, by frequency, intensity, or a composite measure of the two) can lead to different predictions on the effects of increased disturbance, and differently shaped DDRs. We argue this separation of disturbance into different factors may go some way to explain the varied DDRs observed in natural communities.

One prediction of the model considered here is that we expect there to be limited selection pressure for a trade-off between fecundity and defence against disturbance, especially in large communities. This result has some support in the literature: from the data in \cite{martin2010dispersal}, \cite{martin2010divergence} concluded that ``it is unlikely that... survivorship come[s] at the expense of fecundity.'' However, there is some evidence for a fecundity-defence trade-off in edge forests exposed to fire. \cite{williams2012mechanisms} show that after a fire, Australian forest species experience a recruitment boost, yet are more susceptible to fire than the eucalypts they are in competition with, while \cite{hoffmann2009tree} find that savanna species suffer lower levels of topkill than forest species. However, both these studies find there are significant levels of re-sprouting following disturbances, a factor excluded from the current model. Further, topkill may act as a form of disturbance defence, as the individual plant is not killed can can retain its present site after the disturbance. Therefore, while resprouting may appear to support a fecundity-defence trade-off, it may be simply an alternative to maintaining above ground biomass during a fire.

Williams et al state that the key advantage of eucalypts when competing with rainforest species is that they re-sprout at a canopy level, rather than ground level. This allows them to maintain their presence in the forest-edge community, while forest species re-sprouting at ground level will not be able to colonise the area of an eucalyptus individual. This indicates that trade-offs in nature are likely to be complex, and trade-offs may lead to correlations with other life history. For example, wood density has been found to be positively associated with seed size, larger seeds produce denser wood in adult trees \cite{ter2001character}. Seed mass has previously been found to correlate negatively with seed number \citep{turnbull1999seed}, and positively with both defence \citep{niklas1992plant} and plant growth rate \citep{gross1984effects}. However, these results are somewhat equivocal, as wood density has also been reported to correlate with growth rate in a negative manner \citep{king2005tree}. Wood density also correlates slightly negatively with tree size (Wei), while bark thickness, an indicator of disturbance resistance, increases with size. While we consider bark thickness as a measure of fire resistance, this complexity may mean that selection for non disturbance related traits influences bark thickness, and therefore defence. Further, \cite{muller2010tolerance} suggests there is a trade-off between fecundity and tolerance to shade or drought. However, \cite{muller2010tolerance} considers disturbance rate $m$ as a constant, identical across species, and as a separate parameter to species specific tolerance $h_i$, in a heterogeneous environment. Here, in contrast, we consider a homogeneous environment in which species react differently to pulses of increased death rate. While tolerance to constant environmental stress may indeed trade-off against fecundity, we propose that the limited evidence for a trade-off between survival of a pulse-type disturbance, such as fires in edge forest, and fecundity is a consequence of other, more strongly selected trade-offs such as the growth-survival and fecundity-growth trade-offs.

Meanwhile, trade-offs between growth and either fecundity or disturbance resistance (or both) can support multiple species for any system capacity $N$. In a large system, we demonstrate that a trade-off involving all three of the major plant traits will give a probability of randomly selected disturbance resistances supporting two species higher than the probability given by a growth-defence trade-off alone. However, when species responses to disturbance are proportional (such that disturbance intensity for one species is doubled, the intensity for the second species is also doubled), the three dimensional trade-off does not significantly improve the likelihood of coexistence when compared to a fecundity-growth trade-off. These results suggest that the trade-off between fecundity and juvenile growth rate, or competition and colonisation, contributes much more to the maintenance of biodiversity than trade-offs involving disturbance resistance.

We also demonstrate that while the response to disturbance frequency is relatively simple, frequency and intensity can have very different effects on community structure, in accordance with recent theory \citep{miller2011frequency}. We add to this theory by determining analytic conditions for coexistence under given disturbance regimes, and by demonstrating that these conditions are robust to model type, productivity and system size in the case where species differ in juvenile growth rates. In general, this result indicates that it is important to consider the intensity and frequency of disturbance events as separate parameters when studying the effects of disturbance on diversity levels, and emphasise that past disturbances can have long-lasting effects, which persist for decades or centuries, in accordance with \cite{foster1999human}. In the current model, these effects come in the form of persistence of inferior competitor species, which can be sustained for many time steps between disturbances as infrequently as every 400 years. The result here is increased diversity within the community, but disturbance can effect several other aspects of the environment, such as net primary production \citep[e.g.][]{turner2010disturbance}. These results, while arising in a new model, are qualitatively similar to those of \cite{miller2011frequency}, adding weight to the suggestion that the separation of factors controlling disturbance is an important consideration in studying the effects of disturbance on a community. We build on this work by showing that these qualitative results are extremely robust to changes in system capacity and productivity.

We conclude that a trade-off between fecundity, in the form of per capita annual seed production, and juvenile growth rates is more significant in sustaining biodiversity than trade-offs between growth and defence or fecundity and defence, and while species specific reactions to disturbance can slightly improve the likelihood of a fecundity-growth trade-off allowing two species coexistence, this increase is not significant. This concurs with empirical studies, which have found a great deal of support for a trade-off between fecundity and growth rate, or the equivalent competition-colonisation trade-off \citep[e.g.][]{levins1971regional,yu2001competition,tilman1994competition,adler2000space}. The high level of occurrence for this trade-off indicates that it has been an important driver in the evolution of those diverse communities, allowing two or more species to effectively occupy the same resource niche by the different allocation of that resource to their life history traits. That a trade-off between growth and defence can support some coexistence suggests that some support should be found for this trade-off in empirical studies, and this is indeed the case \citep[e.g.][]{wright2010functional,fine2006growth}. However, support for this is much less widespread than the fecundity-growth or competition-colonisation trade-off, which further supports our conclusion that the latter is the more significant driver of biodiversity.

\section*{Acknowledgements}
This work was funded by a NERC studentship for SN. Thanks go to Rob Beardmore and Ben Collen for comments on an earlier version of this manuscript.

\section*{Appendices}

\bappendix
\section{Approximating the expected change function}
\label{app:approximations}
When approximating the average change using the function given by \eqref{avchange}, we must compare these results to the actual expected change. Using the Law of Total Probability ($P(A)=\sum_{b \in B} P(A|b)P(b)$), where $A$ is the probability of increasing species 1 numbers by $k$ individuals, and $B$ the set of possible death numbers in species 2, and summing the results weighted by $k$, the expected change for the model when at the lower boundary $n=1$ is given by
 \begin{align}
\label{acreal1difi}&\text{AverageChangeReal}(1)=\frac{-I_1+(N-1)I_2^{N-1}(1-I_1)}{dNT_D}\\
&+\frac{(1-I_1)\sum_{k=1}^{N-2}\sum_{j=k}^{N-2}k I_2^j(1-I_2)^{N-2-j}{N-2\choose j} Sp_1(1,N-1-j)^k(1-Sp_1(1,N-1-j))^{j-k} {j \choose k}}{dNT_D} \notag \\
 &+ \left(1-\frac{1}{dNT_D}\right)(p_{1,2}-p_{1,0}), \notag
 \end{align}
 while at the upper boundary $n=N-1$, the expected change is given by
\begin{align}
 \label{acrealtotdifi}&\text{AverageChangeReal}(N-1)= \frac{-I_1^{N-1}(N-1) +I_2(1-I_1^{N-1})}{dNT_D} \\
 &-\frac{(1-I_2)\sum_{k=1}^{N-2}\sum_{j=k}^{N-2}k I_1^j(1-I_1)^{N-2-j}{N-2\choose j} Sp_1(N-1-j,1)^{j-k}(1-Sp_1(N-1-j,1))^{k} {j \choose k}}{dNT_D} \notag \\
 &+ \left(1-\frac{1}{dNT_D}\right)(p_{N-1,N}-p_{N-1,N-2}). \notag
  \end{align}
  To reduce the computational power needed, and to gather analytic conditions for coexistence, these functions are each approximated by \eqref{avchange}. The error is calculated for varied times between disturbances $\ln(T_D) \in \{1,2,3,4,5,6,7,8\}$ and different intensities $I_i \in \{0.1,0.2,0.3,0.4,0.5,0.6,0.7,0.8,0.9\}$ for $i=1,2$. In the case of fecundity-growth trade-off, $I_1=I_2$ meant the number of intensity combinations was reduced to 9 for each time between disturbance from 81. Figure~\ref{fig:fdtoapprox} shows how the size of the absolute error between the actual expected change and the approximation declines with system capacity $N$ for a trade-off between fecundity and defence ($x=0$), while Figures~\ref{fig:growthdefenceerrors} and \ref{fig:fecunditygrowtherrors} show that this result holds for the growth-defence trade-off ($s_1=s_2$) and the fecundity-growth trade-off respectively. The approximation is accurate even at relatively small $N>400$. While the results from only one set of parameters are shown, these results are robust to parameter changes.
     \begin{figure}[th]
\centering
   \begin{tabular}{rrrr}
   (a)&&(b)&\\
  &\includegraphics[width=2.5in]{FDmeanerr.pdf} && \includegraphics[width=2.5in]{FDmaxerr.pdf} \\
  (c)&&(d)&\\
  &\includegraphics[width=2.5in]{FDtmeanerr.pdf} && \includegraphics[width=2.5in]{FDtmaxerr.pdf} \end{tabular}
   \caption[Errors in approximating average change: fecundity-defence trade-off]{\textbf{Fecundity-defence trade-off:} (a-b)  How the mean (a) and maximum (b) error size at the lower boundary decrease with increased system capacity $N$. (c-d) How the mean (c) and maximum (d) error size at the upper boundary decrease with increased system capacity $N$. The error was calculated for intensities $I_1,I_2=0.1,0.2,0.3,...,0.9$ and time between disturbances $\ln(T_D)=1,2,3,...,8.$ Parameters used are $s_1=500,s_2=50$. Error is calculated as $| \text{AverageChangeReal}(n) - \text{AverageChange}(n) |$ for $n=1,N-1$.}
    \label{fig:fdtoapprox}
   \end{figure}   
    \begin{figure}[th]
\centering
   \begin{tabular}{rrrr}
   (a)&&(b)&\\
  &\includegraphics[width=2.5in]{GDmeanerr.pdf} && \includegraphics[width=2.5in]{GDmaxerr.pdf} \\
  (c)&&(d)&\\
  &\includegraphics[width=2.5in]{GDtmeanerr.pdf} && \includegraphics[width=2.5in]{GDtmaxerr.pdf} \end{tabular}
   \caption[Errors in approximating average change: growth-defence trade-off]{\textbf{Growth-defence trade-off:} (a-b)  How the mean (a) and maximum (b)  error size at the lower boundary decrease with increased system capacity $N$. (c-d) How the mean (c) and maximum (d) error size at the upper boundary decrease with increased system capacity $N$. The error was calculated for intensities $I_1,I_2=0.1,0.2,0.3,...,0.9$ and time between disturbances $\ln(T_D)=1,2,3,...,8.$ Parameters used are $s_1=50,s_2=50,x=0.06$. Error is calculated as $| \text{AverageChangeReal}(n) - \text{AverageChange}(n) |$ for $n=1,N-1$.}
     \label{fig:growthdefenceerrors}
    \end{figure}
         \begin{figure}[th]
\centering
   \begin{tabular}{rrrr}
   (a)&&(b)&\\
  &\includegraphics[width=2.5in]{meanerrorovermatrix.pdf} && \includegraphics[width=2.5in]{meanerrorovermatrix.pdf} \\
  (c)&&(d)&\\
  &\includegraphics[width=2.5in]{meanerrortotovermatrix.pdf} && \includegraphics[width=2.5in]{maxerrortotovermatrix.pdf} \end{tabular}
   \caption[Errors in approximating average change: fecundity-growth trade-off]{\textbf{Fecundity-growth trade-off:} (a-b)  How the mean (a) and maximum (b) error size at the lower boundary decrease with increased system capacity $N$. (c-d) How the mean (c) and maximum (d) error size at the upper boundary decrease with increased system capacity $N$. The error was calculated for intensities $I_1=I_2=0.1,0.2,0.3,...,0.9$ and time between disturbances $\ln(T_D)=1,2,3,...,8.$ Parameters used are $s_1=500,s_2=50,x=0.06$. Error is calculated as $| \text{AverageChangeReal}(n) - \text{AverageChange}(n) |$ for $n=1,N-1$.}
     \label{fig:fecunditygrowtherrors}
   \end{figure}
   
   \section{Stationary distributions, and the effects of low level immigration}
\label{app:stationary}
In a closed system such as the one detailed above, if there is a non zero probability of reaching a state from which there is zero probability of leaving, the only stable distribution is fixed at one of those absorbing states. On the model considered in the current manuscript, the result of this is that the stable distribution $\Pi$ is given by
\begin{equation}
\pi_i = \begin{cases}
1/2 , i = 0,N \\
0, i = 1,2,\dots,N-1. \end{cases}
\end{equation}
Therefore, any coexistence suggested by invasion analysis is merely temporary. Over ecological timescales of decades or centuries, both species may coexist over several generations,yet at time $t$ tends to infinity, one species will go extinct, although the identity of that species is unknown.

However, when low levels of immigration from outside the system is included, studying the stationary distribution of a system without disturbance demonstrates that the stochastic boundedness conditions are a good approximation of where coexistence is expected to occur (here coexistence is considered to occur when the most probably state in the stationary distribution has both species present).

Compared to this definition of coexistence, we find that stochastic boundedness does slightly enlarge the expected region of coexistence, but generally captures the expected behaviour well. With parameters $s_1=500,s_2=50,N=1000$, we find that when only 10 seeds of each species are expected to arrive in the system each time step, stochastic boundedness conditions adjusted to include this low level of immigration give an expected range of coexistence for $0.015 \lessapprox x \gtrapprox 0.046$.

The stationary distribution is found by solving
\begin{equation}
A \mathbf{\pi} = \mathbf{\pi}
\end{equation}
where $\mathbf{\pi}$ is the stationary distribution such that $\pi_i$ is the probability of having $i-1$ species 1 individuals present. 

The range of $x$ which have an interior maximum is given by $0.02 \lessapprox x \gtrapprox 0.041$. If we consider parameters where an interior peak is present ($\pi_x > \pi_{x-1},\pi_{x+1}$ for some $x \in [1, N-1]$), this range increases to  although the most likely state is still extinction of one species $0.019 \lessapprox x \gtrapprox 0.043$ although at the extremes of this range, the most likely state the system occupies is extinction of one species. This slight overestimation of range by the stochastic boundedness conditions is consistent across parameter changes, Combining this observation with the time series simulations in Figure~\ref{confidenceints}  we conclude that it is an appropriate approximation of the systems behaviour, over ecological timescales in the absence of immigration, and the behaviour as $t \to \infty$ when immigration is considered.
   
   \section{Analytic solutions for fecundity-defence trade-off}
   \label{app:fdtoroots}
    Using Mathematica 8.0.1.0 to set $\text{AverageChange}(1)=0$ and rearranging to solve for $I_2$ gives the solution
 \begin{align}
 &I_2=\\
 &
\frac{ \splitfrac{N^3 s_1 s_2 T_D - 100 N^3 s_1 s_2 - N^3 s_2^2 T_D + 100 N^3 s_2^2 y + N^2 s_1^2
   T_D y - 100 N^2 s_1^2 - N^2 s_1 s_2 T_D y - 3 N^2 s_1 s_2 T_D }{\splitfrac{+100 N^2 s_1 s_2
   y+ 200 N^2 s_1 s_2+ 3 N^2 s_2^2 T_D- 300 N^2 s_2^2 y+ 100 N^2 s_2^2- 2 N
   s_1^2 T_D y- 100 N s_1^2 y+ 100 N s_1^2}{\splitfrac{+2 N s_1 s_2 T_D y+ 2 N s_1 s_2 T_D+ 100
   N s_1 s_2- 2 N s_2^2 T_D+ 200 N s_2^2 y- 300 N s_2^2}{\splitfrac{+ 200 s_1^2 y- 200 s_1
   s_2 y-200 s_1 s_2+200 s_2^2}{\pm \sqrt{ \splitfrac{ \left( \splitfrac{-N^3
   s_1 s_2 T_D + 100 N^3 s_1 s_2+ N^3 s_2^2 T_D- 100 N^3 s_2^2 y- N^2 s_1^2 T_D
   y +100 N^2 s_1^2+ N^2 s_1 s_2 T_D y}{ \splitfrac{ + 3 N^2 s_1 s_2 T_D- 100 N^2 s_1 s_2 y- 200
   N^2 s_1 s_2- 3 N^2 s_2^2 T_D+ 300 N^2 s_2^2 y- 100 N^2 s_2^2}{ \splitfrac{+ 2 N s_1^2 T_D y+ 100 N  s_1^2 y- 100 N s_1^2-2 N s_1 s_2 T_D y-2 N s_1 s_2 T_D-100 N s_1
   s_2}{+2 N s_2^2 T_D-200 N s_2^2 y+300 N s_2^2-200 s_1^2 y+200 s_1 s_2
   y+200 s_1 s_2-200 s_2^2}}} \right)^2}{\splitfrac{-4 \left(\splitfrac{ N^3 s_1 s_2 T_D-N^3
  s_2^2 T_D+N^2 s_1^2 T_D-4 N^2 s_1 s_2 T_D-100 N^2 s_1 s_2+3 N^2
   s_2^2 T_D}{\splitfrac{+100 N^2 s_2^2-2 N s_1^2 T_D-100 N s_1^2+4 N s_1 s_2 T_D+400 N s_1
   s_2-2 N s_2^2 T_D}{-300 N s_2^2+200 s_1^2-400 s_1 s_2+200 s_2^2}}\right)}
   {\times \left(\splitfrac{-100 N^3 s_1 s_2 y+100 N^3 s_2^2 y-100 N^2 s_1^2 y}{\splitfrac{+400 N^2 s_1 s_2
   y-300 N^2 s_2^2 y}{+100 N s_1^2 y-300 N s_1 s_2 y+200 N s_2^2 y}} \right)}}}}}}}}   {2 \left(-100 N^3 s_1 s_2 y+100 N^3 s_2^2
   y-100 N^2 s_1^2 y+400 N^2 s_1 s_2 y-300 N^2 s_2^2 y+100 N s_1^2 y-300 N s_1
   s_2 y+200 N s_2^2 y \right)} \notag
   \end{align}
   while $\text{AverageChange}(N-1)=0$ returns
   \begin{align}
   &I_2=\\
   & \frac{ \splitfrac{N^3 s_1^2 T_D y- 100 N^3 s_1^2- N^3 s_1 s_2 T_D y+ 100 N^3 s_1 s_2 y- 3 N^2
   s_1^2 T_D y -100 N^2 s_1^2 y+ 300 N^2 s_1^2+ 3 N^2 s_1 s_2 T_D y }{ \splitfrac{+N^2 s_1 s_2
   T_D-200 N^2 s_1 s_2 y- 100 N^2 s_1 s_2- N^2 s_2^2 T_D+ 100 N^2 s_2^2
   y+ 2 N s_1^2 T_D
   y+ 300 N s_1^2 y- 200 N s_1^2}{\splitfrac{-2N s_1 s_2 T_D y - 2 N s_1 s_2 T_D- 100 N s_1
   s_2 y+ 2 N s_2^2 T_D- 100 N s_2^2 y}{\splitfrac{+ 100 N s_2^2- 200 s_1^2 y+ 200 s_1 s_2
   y +200 s_1 s_2-200 s_2^2}{ \pm \sqrt{\splitfrac{ \left( \splitfrac{-N^3 s_1^2 T_D y+100 N^3 s_1^2+N^3 s_1 s_2 T_D y-100 N^3 s_1 s_2
   y+3 N^2 s_1^2 T_D y+100 N^2 s_1^2 y}{\splitfrac{-300 N^2 s_1^2-3 N^2 s_1 s_2 T_D y- N^2 s_1
   s_2 T_D+ 200 N^2 s_1 s_2 y+ 100 N^2 s_1 s_2+ N^2 s_2^2 T_D}{\splitfrac{- 100 N^2 s_2^2 y- 2
   N s_1^2 T_D y-300 N s_1^2 y+200 N s_1^2+2 N s_1 s_2 T_D y+2 N s_1 s_2
   T_D}{\splitfrac{+100 N s_1 s_2 y-2 N s_2^2 T_D+100 N s_2^2 y-100 N s_2^2}{+200 s_1^2 y-200
   s_1 s_2 y-200 s_1 s_2+200 s_2^2}}}}\right)^2}{\splitfrac{-4 \left(\splitfrac{N^3 s_1^2 T_D-N^3 s_1
   s_2 T_D-3 N^2 s_1^2 T_D-100 N^2 s_1^2+4 N^2 s_1 s_2 T_D+100 N^2 s_1 s_2}{\splitfrac{-N^2
   s_2^2 T_D+2 N s_1^2 T_D+300 N s_1^2-4 N s_1 s_2 T_D-400 N s_1 s_2}{+2 N
   s_2^2 T_D+100 N s_2^2-200 s_1^2+400 s_1 s_2-200 s_2^2}}\right)}{ \times \left(\splitfrac{-100 N^3
   s_1^2 y+100 N^3 s_1 s_2 y+300 N^2 s_1^2 y-400 N^2 s_1 s_2 y}{+100 N^2
   s_2^2 y-200 N s_1^2 y+300 N s_1 s_2 y-100 N s_2^2 y}\right)}}}}}}}}
   {2 \left(-100 N^3 s_1^2 y+100 N^3 s_1 s_2 y+300 N^2
   s_1^2 y-400 N^2 s_1 s_2 y+100 N^2 s_2^2 y-200 N s_1^2 y+300 N s_1 s_2
   y-100 N s_2^2 y\right)}.   \notag
   \end{align}
   Note that each of these solutions takes to form
   $$
  \frac{ a\pm \sqrt{b}}{c}
  $$
  and therefore there are actually two solutions to each equation. However, in each case, one one of the roots falls within the range of possible intensities $[0,\min(1/y,1)]$. In each case, the positive square root gives a solution outside this range, and we therefore use the negative square roots to determine the region of coexistence.
   
   \section{Comparing simulations to analytic results}
   \label{app:simcompare}
   We ran a range of time series simulations to verify the predictions given by our analytical results. Simulations are run at $I=0.1,0.2,0.3,0.4,0.5,0.6,0.7,0.8,0.9$ and $1$ and frequencies $\ln(T_D)=1,2,3,4,5,6,7,8$. Each point has 20 simulations run for $10^5$ time steps, and a linear interpolation algorithm is used between data points. Dark colours in Figure~\ref{fig:simulationdata}(a) are indicative of high probability of coexistence, where both species persist in the community after $10^5$ time steps. The scale to the right shows the percentage of the 20 time series exhibit coexistence. This data indicates that while the outcomes are accurately predicted by the analysis when disturbance events are very frequent (Figure~\ref{fig:simulationdata}(b)), there is a ``tail'' at high intensity and intermediate frequency where coexistence is only possible for a very narrow range of frequencies. However, frequencies that are sufficiently high (such that the lines given by $\text{AverageChange}(1)=0$ and $\text{AverageChange}(N-1)=0$  are approximately constant in $\ln(T_D)$) the average change function provides a good fit to the simulation data (Figure~\ref{fig:simulationdata}). This is also the case for simulations in a system of size $N=10000$, where the average change provides a good fit for frequencies where the region of coexistence is approximately constant in $f$ (results not shown).
\begin{figure}\begin{tabular}{ll}
(a)&(b)\\
 \includegraphics[width=2.5in]{simcoexist.pdf}& \includegraphics[width=2.5in]{hockeyTd.pdf}\end{tabular}
   \caption[Comparing simulation data with analytical predictions]{\textbf{Comparing simulation data with analytical predictions:} (a) Simulation data showing how stochasticity effects the predicted region of coexistence at high intensity. 20 time series simulations were run for each point. The darker the shading, the greater the number of these time series that exhibited coexistence after $10^5$ time steps. Linear scaling is performed between data points. (b) The region of coexistence predicted by invasion analysis. Parameters $s_1=500,s_2=50,x=0.06,N=1000$}
 \label{fig:simulationdata}
\end{figure}

\bibliographystyle{spbasic}
\bibliography{Papers.bib}
   
\end{document}